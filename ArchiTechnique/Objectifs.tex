\section{Objectifs fonctionnels}

\subsection{Présentation générale}

\paragraph{} Nous souhaitons mettre en place un environnement de travail basé
sur l'utilisation de postes virtuels à destination des étudiants et des
enseignants du département informatique de l'INSA de Lyon.

\paragraph{} L'utilisation de machines virtuelles offre de nombreux avantages
pour les utilisateurs :

\begin{description}
  \item[Flexibilité] Les utilisateurs peuvent être administrateurs de leur
  environnement : ils peuvent choisir leur système d'exploitation et le
  maintenir comme ils le souhaitent, à l'image de leur ordinateur personnel.
  Par ailleurs, les équipes pédagogiques peuvent mettre en place des
  environnements qui correspondent exactement aux besoins des travaux pratiques
  sans risquer de compromettre l'intégrité de l'environnement, et donc de
  provoquer des conflits avec d'autres outils.
  \item[Sécurité] La virtualisation permet aux administrateurs de contrôler
  l'allocation des ressources matérielles. Les différents postes sont exécutés
  dans des cages (Sandbox), et ne peuvent donc pas s'en prendre directement au
  matériel.\\
  On notera que l'accès aux répertoires partagés peut être limités en
  faveur de la mise à disposition des documents directement dans les machines
  virtuelles configurées par les enseignants. Par ailleurs, le partage de
  fichiers offre, avec cette solution, le même niveau de sécurité que l'accès
  aux documents par les utilisateurs depuis l'extérieur du réseau INSA via
  SFTP.\\
  Enfin, les utilisateurs peuvent conserver le contrôle intégral des données
  qu'ils manipulent sur les machines virtuelles.
  \item[Portabilité] Les étudiants ayant besoin de travailler à distance
  peuvent télécharger une copie des machines virtuelles et les exécuter sur
  leur ordinateur personnel en utilisant un hyperviseur compatible. Par
  ailleurs, la diversité du matériel est lissée par l'hyperviseur.
  \item[Facilité d'utilisation] Les outils d'administration mis à disposition
  de la plupart des hyperviseurs offrent des utilitaires graphiques accessibles.
  \item[Amélioration de la maintenance] La maintenance de l'environnement est
  simplifiée à moyen terme. En effet, une machine virtuelle abimée peut être
  restaurée rapidement à son état initial grâce aux sauvegardes.\\
  Par ailleurs, la procédure de sauvegarde est uniforme (on sauvegarde l'image
  de la machine, peu importe son système d'exploitation).\\
  Enfin, l'infrastructure supportant des hyperviseurs bare-metal en réseau
  peut être mise à jour de manière pratiquement transparente : on installe un
  nouveau serveur d'exécution et l'ajoute au réseau. Un serveur ordonnanceur
  pourra alors utiliser directement sa puissance de calcul pour exécuter les
  machines virtuelles.
  \item[Performance] Malgré le coût de la virtualisation en terme de
  ressources, il est possible d'abstraire la quantité de ressource
  (calcul, mémoire) disponible pour une machine virtuelle de la quantité de
  ressource effectivement mise en place dans le matériel. En effet, il est
  possible d'exécuter une machine virtuelle sur plusieurs serveurs, et donc de
  mettre facilement en place des solutions de calcul distribué.
\end{description}

\subsection{Environnement utilisateur}

\paragraph{} Les utilisateurs ont accès, depuis n'importe quel poste mis à sa
disposition, à un ensemble d'images de machines virtuelles qu'il peut exécuter.
Il est libre de créer, modifier et supprimer les caractéristiques de ses
machines virtuelles.

\paragraph{} Chaque utilisateur dispose d'un quota de ressource qu'il peut
utiliser librement. Par exemple, on peut mettre à disposition de chaque
utilisateur l'équivalent d'un coeur (processeur) cadencé à 1,6Ghz, 2Go de
mémoire vive et 15 Go de stockage qu'il est libre d'utiliser comme il le
souhaite.

\paragraph{} Lorsqu'un utilisateur accède à un poste, il s'authentifie et
accède à la liste des machines qu'il peut lancer. Enfin, il peut lancer une
machine virtuelle sur le poste.

\subsection{Services à disposition des enseignants}

\paragraph{} Les enseignants disposent d'outils supplémentaires. Ils peuvent
notamment créer des images de machines virtuelles qui peuvent être dupliquées
et affectées aux étudiants.

\paragraph{} Les enseignants peuvent donner des droits limités aux étudiants
(ils ne peuvent, par exemple, pas contrôler le quota de ressources disponibles)
sur ces machines et peuvent toujours y accéder.

\paragraph{} Les enseignants peuvent également retirer les accès aux étudiants,
ou déclencher des sauvegardes forcées, par exemple pour fixer la date limite de rendu des projets.

\subsection{Services à disposition des administrateurs}

\paragraph{} Les administrateurs disposent d'un ensemble d'outils permettant
d'administrer le système.

\begin{description}
  \item[Monitoring] L'état des serveurs et des machines virtuelles est contrôlé
  en temps réel. Les administrateurs peuvent contrôler la charge des serveurs,
  le nombre de machines lancées, le trafic sur le réseau, etc.
  \item[Administration des machines] Les administrateurs contrôlent les droits
  et les propriétés des machines virtuelles à disposition des utilisateurs.
  \item[Gestion des utilisateurs] Les administrateurs peuvent contrôler les
  groupes et les droits des utilisateurs ayant accès au système.
  \item[Contrôle des paramètres du système] Les administrateurs peuvent,
  notamment, paramétrer les liens avec les les annuaires d'utilisateurs.
  \item[Gestion de modèles de machines virtuelles] Les administrateurs peuvent
  mettre à disposition des images standards de machines, permettant aux
  utilisateurs ne souhaitant pas configurer leur machine de déployer rapidement
  une machine virtuelle pré-configurée par les administrateurs.
\end{description}

\subsection{Outils de monitoring}

\paragraph{} Sur les systèmes d'exploitation supportés, il est possible
d'installer des outils permettant aux enseignants de contrôler l'état des
machines, par exemple pour vérifier le résultat d'un TP.

\subsection{Partage des documents}

\paragraph{} Il est possible de configurer les machines virtuelles pour mettre
en place des points de montage liés à un système de fichier partagé par
l'ensemble des machines virtuelles. Ceci peut, par exemple, permettre de
partager un répertoire utilisateur (ou groupe) sur plusieurs machines.
