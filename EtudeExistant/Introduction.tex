Dans ce document, nous allons étudier les systèmes et solutions techniques
existantes pouvant correspondre au projet de plateforme pédagogique que nous
souhaitons présenter.

Nous souhaitons proposer une plateforme permettant d'utiliser des machines
virtuelles de manière simple et flexible aux étudiants et enseignants d'un
département informatique. L'objectif est double. D'une part, nous souhaitons
permettre aux étudiants d'utiliser une machine virtuelle intégralement
configurable à la place d'une session Windows ou GNU/Linux aux droits limités
sur des postes partagés. D'autre part, les enseignants seront en mesure
d'installer et configurer des machines virtuelles adaptées aux travaux pratiques
qu'ils souhaitent réaliser, pourront les distribuer aux étudiants et mettre en
place des outils d'évaluation pour les étudiants directement dans les machines.

Nous allons essentiellement présenter les solutions de virtualisation
existantes, qui connaissent un essor depuis plusieurs années.
