\section{Etat de l'art de la virtualisation de postes de travail}

\subsection{Virtualisation ?}

\subsubsection{Qu'est ce que c'est ?}

Virtualiser une machine, reviens à exécuter un ou plusieurs systèmes d'exploitation dans un contexte dit virtuel. Ce qui signifie que l'OS n'a pas la main directement sur le matériel, mais via une couche d'abstraction intermédiaire.\\ 

Le principe de fonctionnement est relativement simple : \\
\begin{itemize}
\item  Une machine hote, fonctionnant à l'aide de n'importe quel système d'exploitation héberge un logiciel de virtualisation .
\item Ce logiciel de virtualisaiton manipule des images de machines virtuelles, et les exécutes dans un contexte virtuel différent, simplifiant la cohabitation entre les différents systèmes d'exploitation.
\item Ces machines possèdent les mêmes caractéristiques qu'une machine physique, des ressources de la machines hôte lui sont attribuées (ie, une quantité de mémoire fixe,  des coeurs du processeur...)
\item Via le logiciel de virtualisation, les machines exécutées peuvent accéder aux ressources de la machine physique, par exemple une imprimante raccordée en usb... etc...
\item Il est alors possible de manipuler facilement une machine virtuelle via son image, elle devient exécutable sur n'importe quel poste exécutant un logiciel de virtualisation. De plus, les opérations de clonage, de sauvegarde / restauration deviennent extrêmement simples.
\end{itemize}

\subsubsection{Technologies}

\begin{itemize}
\item Noyau partagé \& Isolateur : Cette solution n'est pas à proprement parler de la virtualisation, car il s'agit juste d'isoler des processus de façon à pouvoir permettre leur exécution simultanée, alors qu'ils ne sont pas prévus pour fonctionner simultanément. Les performances sont bonnes, car peu d'overhead est généré à l'exécution, mais cela est réservé aux systèmes Linux uniquement, car le procédé est extrêmement dépendant du système.
\item Noyau en User Space : Cela reviens à exécuter le noyau hébergé dans l'user space de la machine hôte, sans abstraction ni optimisation au niveau matériel. La encore réservée aux systèmes linux, cette solution se révelle peu performante et difficile à mettre en place.
\item Émulation matérielle  : dernière solution, elle consiste à créer une couche d'émulation matérielle entre la machine hôte et la machine hébergée, afin de gérer et d'optimiser les interactions avec le matériel. Le type de logiciel utilisé pour exécuter ces machines virtuelles est appelé hyperviseur.
\end{itemize}

\subsubsection{Avantages ?}

A première vue, l'introduction d'une couche intermédiaire entre l'OS et le matériel peut être vue comme une source de complexité supplémentaire et donc un overhead certain. Cela est de moins en moins vrai compte tenu des différentes technologies d'hyperviseurs, l'intégration et l'optimisation de ces derniers allant jusqu'à la microprogramation des processeurs actuels.\\
Passé ce léger défaut, cela contribue à la mutualisation des ressources d'exécution et facilite l'équilibrage de charge. Par exemple, la mutualisation de quatre serveurs sur une seule machine physique peut se révéler intéressante en terme de coût et de consommation énergétique.\\
De plus, l'aspect dématérialise de ce type de machine permet de rendre beaucoup plus facile l'administration et la manipulation de ces dernières.

\subsection{Hyperviseur ?}

Comme énoncé un peu plus haut, un hyperviseur est un logiciel d'exécution de machines virtuelles. L'objectif de ce logiciel est dans un premier temps de créer une couche d'abstraction matérielle entre lui et l'OS hébergé, et ensuite de gérer les ressources de la machine hôte en les allouant aux différentes machines virtuelles qu'il exécute en fonction du besoin.\\
D'autre fonctionnalités sont alloués, comme la gestion de l'accès aux ressources physique de la machine hôte, ou encore les backups des différentes machines virtuelles exécutées.\\
On distingues deux catégories d'hypervieurs :\\

\subsubsection{Hyperviseur "Bare Metal"}

Ces hyperviseurs intégrent le système d'exploitation de l'hote. Ce sont généralement des systèmes ultra-légers et optimisés de façon. Ainsi, les pertes de performances sont limités, l'hyperviseur Bare Metal limitant grandement l'overhead nécessaire à l'exploitation de la machine. Cependant, la machine exécutant cet OS est donc dédiée aux tâches de virtualisation.

\subsubsection{Hyperviseur "Hosted"}

A l'inverse, l'hyperviseur est ici exécuté dans l'user space d'un OS standard. Les performances dédiées aux VM sont donc moins importantes, mais la machine hôte n'est pas dédiée à la virtualisation, ce qui est très pratique pour faire cohabiter des systèmes d'exploitation sur un ordinateur, et plus accessible qu'une solution de dual-boot standard.

\subsection{Déploiement au sein d'une architecture réseau}

Intéressons nous maintenant d'un peu plus prêt à l'architecture d'un système de virtualisation et à son implémentation dans un SI.

\subsubsection{Serveurs}

Afin d'exploiter un système de virtualisation globalisé à un SI, plusieurs fonctionnalités principales sont nécessaires :\\
\begin{itemize}
\item Exécuter des VM sur des ressources partagées
\item Stocker et synchroniser le parc de VM
\item Administrer le parc de VM
\item Répartir la charge sur les différents hyperviseurs bare-metal.
\item Administrer le parc utilisateur et gérer les droits\\
\end{itemize} 

Chacune de ces quatre fonctionalités est couverte par un serveur autonome, dans une architecuture de VM nous retrouverons donc :\\

\paragraph{Grappe de serveurs d'exécution} Serveurs exécutant des hyperviseurs de type bare-metal,  ayant pour objectif de fournir une ressource matérielle dédiée à l'exécution des machines virtuelles. Ils sont administrés à distance, la charge étant répartie par un autre serveur. 

\paragraph{Serveur de Stockage / Synchronisation des données} Ces serveurs sont en charge de stocker les images de machines virtuelles et de garantir le backup de ces dernières pendant leur exécution.

\paragraph{Serveur d'administration} Serveur, offrant un service de pilotage et de monitoring de l'activité du service de virtualisation. En général, c'est un service web, accessible à plusieurs classes d'utilisateurs, ayant des accréditation adaptées en fonction de leurs droits.

\paragraph{Serveur de répartition de charge}
"Ordonnanceur" du système de virtualisation, il prends en charge les demandes distanciation de VM et réparti ces demandes sur les ressources d'exécution qu'il tient à sa disposition.

\paragraph{Serveur d'authentification} 
Point d'entrée au service, ce serveur se charge de vérifier l'identité des différents clients effectuant une demande de connexion au service de VM, via par exemple le raccordement à un annuaire LDAP.

\subsubsection{Postes de travail physiques}

Les postes de travails physiques exécutent un client qui se connecte au système de virtualisation. Il intègre tout les utilitaires nécessaires à l'accès aux ressources physique de la machine hébergeant le client, pour donner l'impression à l'utilisateur final que la VM s'exécute comme OS principal de la machine.\\
Deux stratégies quand à l'exécution des machines virtuelles :\\
\begin{itemize}
\item Exécution distante (sur un hyperviseur bare-metal) et accès via un protocole de bureau à distance, type VNC ou encore protocole propriétaire PCoIP (VWare...) . Centralise l'exécution et réparti efficacement la charge.
\item Exécution sur la machine client, utilisant alors le client comme hyperviseur dédidé. L'image de la machine est donc rapatriée depuis le serveur de stockage pour être exécutée, et backupée à intervalle régulier.
\end{itemize}

Dans notre cas, il est évident que l'utilisation du mode d'exécution local se révelle extrêmement intéressant, car il permettrait d'exploiter la totalité de la puissance de calcul des postes du département informatique.

\subsubsection{Politique de gestion utilisateurs}

Nous le verrons par la suite, beaucoup de services de virtualisation propose un lien étroit avec les systèmes d'annuaires LDAP.

\subsection{Solutions existantes}

\subsubsection{Citrix XenServer}

\subsubsection{Solution VMWare}

L'éditeur logiciel VMWare propose une solution répondant à nos besoins. La solution intégrée proposée couvre la totalité des besoins générés par l'adoption de ce type de technologie. La couverture de tout les logiciels d'exploitation par l'éditeur permet d'exploiter de nombreuses synergies entre ces différents composants, par exemple par la mise en place de protocoles propriétaires optimisés... Nous allons passer en revue les points clés de notre besoin.\\

\paragraph{Hébergement centralisé de machines virtuelles}

\textbf{VMWare ESXi} est un hyperviseur de type bare-metal, reconnu comme l'un des meilleurs du marché actuellement. L'avantage de ce type de solution à OS ultra léger est double, d'une part l'optimisation du fonctionnement de l'hyperviseur exploitant au maximum les ressources du serveur, d'autre part une plus grande sécurité sur cette plateforme.

\paragraph{Équilibrage de charge sur les différents serveurs}

\textbf{VMWare vCenter Server} offre une solution de supervision de l'activité du parc de VM en permettant d'allouer dynamiquement les VM en activité sur le parc de serveur, ainsi nous sommes surs de tirer parti au maximum des ressources d'exécution des serveurs, et ce sans interruption de service pour l'utilisateur. Nous envisageons cette solution pour les machines dites "partagées" uniquement, par exemple les serveurs de TP entre autres...

\paragraph{Client d'accès dédié}
Le client VMWare View fourni un moyen flexible d'exploiter les différentes machines virtuelles du système.\\
Tout d'abord, il est multiplateforme, (Windows, Linux, MacOS, Android, autres...) et cela se révèle extrêmement intéressant en terme de limitation de coûts, car il serait alors envisageable de migrer la totalité du parc de machines physiques  sur un système libre, l'utilisateur final étant parfaitement libre de choisir son OS sur sa VM.\\ 
Pour continuer, il est possible de rapatrier et d'exécuter sur la machine hébergeant le client les machines virtuelles, permettant d'exploiter le parc de machines physiques présentes localement. Cela peut s'avérer extrêmement intéressant la encore, pour exploiter l'existant technique (ie le parc de machines) sans avoir besoin de renouveler les serveurs... Car exécuter de façon distante les VM impliquerais un renouvellement du parc de serveurs.\\

\paragraph{Administration facilitée du parc de VM}
\textbf{VMWare View Composer} Cette solution permet de générer un pool de machines virtuelles "filles" héritant d'une seule et même machine mère. Ainsi le déploiement et l'administration pour par exemple une plateforme de TP est grandement simplifiée. D'autant plus que les coûts de stockages sont réduits à environ 60\% d'après les dires du constructeur.\\
De plus, une plateforme web...%TODO

\paragraph{Gestion des accès utilisateurs}
Tout les composants de cette suite logicielle sont directement racordable à un annuaire LDAP, plus particulièrement Microsoft Active Directory, qui est actuellement utilisé pour le SI de l'école. Ainsi, nous serions en mesure de fournir une gestion fine des droits utilisateurs sur le système de machines virtuelles, par exemple au niveau de la création /administration de ces dernières, en se basant sur l'arbre de classes actuellement défini par l'annuaire de l'école.

\subsubsection{Solution Microsoft Hyper-V}

\subsubsection{Proxmox \& solutions Open-Source}


\subsection{Evaluation des solutions et choix}