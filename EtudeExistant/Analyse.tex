
\section{Etat de l'art de la virtualisation de postes de travail}

\subsection{Virtualisation ?}

\subsubsection{Qu'est ce que c'est ?}

\paragraph{} Virtualiser une machine revient à exécuter un ou plusieurs
systèmes d'exploitation dans un contexte dit virtuel. Le matériel constituant la
machine virtuelle est émulé par un logiciel chargé d'être l'intermédiaire entre
l'ordinateur physique (l'hôte) et le système virtuel (la cible).

\paragraph{} Le principe de fonctionnement est relativement simple :
\begin{itemize}
\item  On installe sur une machine hôte, un logiciel de virtualisation. Celui-ci
peut fonctionner sur un système d'exploitation existant ou être directement
intégré à un système d'exploitation dédié.
\item Ce logiciel de virtualisation manipule des images de machines virtuelles,
qui définissent le matériel virtuel de la machine cible. Généralement, c'est
images sont liés à un disque dur virtuel, qui contient le système d'exploitation
de la machine virtuelle. Une image de machine est exécutée dans un contexte qui
lui est propre. Cela signifie que la machine virtuelle dispose, notamment, de
son propre espace mémoire. Ainsi, le système d'exploitation de la machine hôte
et ceux des machines cibles fonctionnent de manière autonomes. Ceci permet, au
final, d'exécuter n'importe quel système d'exploitation cible, même si, dans
certains cas, ce système d'exploitation est conçu pour du matériel complètement
différent de celui de l'ordinateur hôte.
\item On dotera généralement les machines virtuelles des mêmes caractéristiques
qu'une machine physique. Le système d'exploitation cible n'aura généralement pas
"conscience" qu'il est exécuté dans un environnement virtuel. Ainsi, on attribue
des ressources de la machines hôte : une quantité de mémoire fixe, un quota
d'accès aux coeurs des processeurs.
\item Un logiciel de virtualisation permet généralement de contrôler l'accès aux
périphériques (réseau, USB, etc), ainsi les machines exécutées peuvent accéder
aux ressources à disposition de l'hôte.
\item Il est alors possible de manipuler facilement une machine virtuelle via
son image, elle devient exécutable sur n'importe quel poste exécutant un
logiciel de virtualisation. De plus, les opérations de clonage, de sauvegarde et
restauration deviennent généralement simples.
\end{itemize}

\subsubsection{Technologies}

\paragraph{} La virtualisation d'un système peut s'effectuer à plusieurs niveaux
: on peut vouloir émuler le matériel ou simplement cloisonner l'exécution de
certains processus (Sandboxing).

\begin{itemize}
\item Noyau partagé \& Isolateur : cette solution de virtualisation a été conçue
pour le noyau Linux. Il s'agit ici seulement d'isoler des processus de façon à
pouvoir permettre leur exécution simultanée sans qu'ils puissent entrer en
concurrence. Chaque appel système effectué par un processus isolé est filtré et
potentiellement réécrit pour ne pas pénaliser le système. Cette solution offre
des bonnes performances, car le surcoût à l'exécution (overhead) est limité : le
noyau n'est chargé qu'une fois en mémoire et le matériel est sollicité
directement. En contrepartie, il n'est pas possible d'exécuter plusieurs
systèmes d'exploitation simultanément.
\item Noyau en User Space : Cela reviens à exécuter le noyau hébergé dans l'user
space de la machine hôte, c'est à dire exécuter un noyau Linux comme un
programme classique sans abstraction du matériel. Cette solution n'a été
implémentée que sur un système Linux, et n'est pas réellement exploitée car elle
se révèle peu performante et difficile à mettre en place.
\item Émulation matérielle : Cette troisième solution consiste à créer une
couche d'émulation matérielle entre la machine hôte et la machine hébergée, afin
de créer un environnement matériel virtuel. Le type de logiciel utilisé
permettant d'exécuter ces machines virtuelles est appelé hyperviseur. Si le type
de matériel de la machine hébergée et de la machine hôte est compatible, il est
possible d'exploiter le matériel sans véritable émulation logicielle. Des
constructeurs tels qu'Intel proposent des processeurs capables d'exécuter des
instructions dédiées à la virtualisation.
\end{itemize}

\subsubsection{Avantages ?}

\paragraph{} À première vue, l'introduction d'une couche intermédiaire entre
l'OS et le matériel peut être vue comme une source de complexité supplémentaire,
une réduction des performances et donc d'un surcout (overhead) certain.
La virtualisation est aujourd'hui un enjeu industriel qui a conduit au
développement de différentes technologies d'hyperviseurs, qui l'intègrent
aujourd'hui des optimisations matérielles très puissantes, allant jusqu'à la
microprogrammation des processeurs actuels.

\paragraph{} Au delà du léger impact sur la performance, la virtualisation
contribue à la mutualisation des ressources d'exécution et facilite
l'équilibrage de charge. Par exemple, la mutualisation de quatre serveurs sur
une seule machine physique peut se révéler intéressante en terme de coût et de
consommation énergétique.

\paragraph{} De plus, l'aspect dématérialisé de ce type de machine permet de
rendre beaucoup plus facile l'administration et la manipulation de ces
dernières.

\subsection{Hyperviseur ?}

\paragraph{} Comme énoncé un peu plus haut, un hyperviseur est un logiciel
d'exécution de machines virtuelles. L'objectif de ce logiciel est dans un
premier temps de créer une couche d'abstraction matérielle entre le système hôte
et l'OS hébergé, et ensuite de gérer les ressources de la machine hôte en les
allouant aux différentes machines virtuelles qu'il exécute en fonction des
besoins.

\paragraph{} En plus de ces fonctionnalités fondamentales, les hyperviseurs
proposent des souvent des outils complémentaires, comme la gestion de l'accès aux
ressources physiques de la machine hôte, ou encore les sauvegardes (backups) des
différentes machines virtuelles exécutées.

\paragraph{} On distingues deux catégories d'hypervieurs.

\subsubsection{Hyperviseur "Bare Metal"}

\paragraph{} Ces hyperviseurs intégrent le système d'exploitation de l'hôte. Ce
sont généralement des systèmes ultra-légers et optimisés pour la virtualisation.
Ainsi, les pertes de performances sont totalement limités, car l'hyperviseur
Bare Metal réduit grandement le surcoût nécessaire à l'exploitation de la
machine. Cependant, la machine hôte exécutant cet OS est nécessairement dédiée
aux tâches de virtualisation.

\subsubsection{Hyperviseur "Hosted"}

\paragraph{} À l'inverse, l'hyperviseur est ici exécuté comme un programme
utilisateur sur un système d'exploitation quelconque. Les performances dédiées
aux machines virtuelles sont moins importantes (l'hyperviseur n'ayant pas
nécessairement une priorité sur les autres programmes), mais cette solution
permet notamment de déployer des machines virtuelles sur un poste de travail
classique, et donc de faire cohabiter des systèmes d'exploitation sur un
seul ordinateur de manière plus flexible qu'avec une solution de dual-boot.

\subsection{Déploiement au sein d'une architecture réseau}

\paragraph{} Intéressons nous maintenant d'un peu plus prêt à l'architecture
d'un système de virtualisation et à son implémentation dans un SI.

\subsubsection{Serveurs}

\paragraph{} Afin d'exploiter un système de virtualisation globalisé à un SI,
plusieurs fonctionnalités principales sont nécessaires :

\begin{itemize}
\item Exécuter des VM (Virtual Machines, ou Machines virtuelles) sur des
ressources partagées,
\item Stocker et synchroniser le parc de VM,
\item Administrer le parc de VM,
\item Répartir la charge sur les différents hyperviseurs bare-metal,
\item Administrer le parc utilisateur et gérer les droits.
\end{itemize} 

\paragraph{} Chacune de ces quatre fonctionalités peut être couverte par un serveur
autonome, dans une architecuture de VM nous retrouverons donc :

\paragraph{Grappe de serveurs d'exécution} Serveurs exécutant des hyperviseurs
de type bare-metal,  ayant pour objectif de fournir une ressource matérielle
dédiée à l'exécution des machines virtuelles. Ils sont administrés à distance,
la charge étant répartie par un autre serveur (le Scheduler, ou Ordonnanceur). 

\paragraph{Serveur de Stockage, ou de Synchronisation des données} Ces serveurs
sont en charge de stocker les images de machines virtuelles et de garantir leur
sauvegarde pendant leur exécution.

\paragraph{Serveur d'administration} Serveur offrant un service de pilotage et
de monitoring de l'activité du service de virtualisation. En général, c'est un
service web, accessible à plusieurs classes d'utilisateurs, ayant des
accréditations adaptées en fonction de leurs droits.

\paragraph{Serveur de répartition de charge} Ordonnanceur du système de
virtualisation, il prends en charge les demandes d'instanciation (création et
démarrage) des VM et répartit les demandes sur les ressources d'exécution qu'il
tient à sa disposition.

\paragraph{Serveur d'authentification} Ce serveur gère l'annuaire des
utilisateurs et se charge de vérifier l'identité des différents clients
effectuant une demande de connexion au service de VM, via par exemple le
raccordement à un annuaire LDAP.

\subsubsection{Postes de travail physiques}

\paragraph{} Les postes de travail physiques exécutent un client qui se connecte
au système de virtualisation. Il intègre tout les logiciels nécessaires à
l'accès aux ressources physiques de la machine hébergeant le client, rendant
ainsi l'exécution de la machine virtuelle transparente.

\paragraph{} Nous pouvons évoquer deux stratégies quand à l'exécution des
machines virtuelles :

\begin{itemize}
\item L'exécution de la machine virtuelle se déroule sur un serveur distant (sur
un hyperviseur bare-metal) et le client à accède via un protocole de bureau à
distance, type VNC, PCoIP (Protocole propriétaire développé par VWare), ou à
l'aide de boitiers KVM (Keyboard, Video, Mouse) permettant de multiplexer sur le
réseau ethernet la communication entre le clavier, la souris, l'écran et l'unité
centrale. L'exécution des VM est centralisé et la charge est répartie par
l'ordonnanceur.
\item L'exécution s'effectue sur le poste client, utilisant alors le client
comme hyperviseur dédidé. L'image de la machine est téléchargée depuis le
serveur de stockage pour être exécutée, et régulièrement sauvegardée.
\end{itemize}

\paragraph{} Les deux solutions sont pratiquement antagonistes : leurs avantages
et inconvénients respectifs s'opposent pratiquement. En effet, l'exécution
sur le poste client est globalement plus souple, répartit la charge de
l'exécution des machines virtuelles et s'affranchit de la question du contrôle
de l'environnement à distance.

\paragraph{} Cependant, cette solution devient inutilisable à partir d'un
certain volume de postes. Il devient nécessaire de disposer de postes clients
suffisament performants d'une part, et d'autre part, le transfert des images de
machines virtuelles solliciterai beaucoup trop le réseau lors de leur lancement.
Il faudrait alors plusieurs minutes pour qu'un utilisateur puisse télécharger et
exécuter son environement de travail.

\subsubsection{Politique de gestion utilisateurs}

\paragraph{} Nous le verrons par la suite, beaucoup de services de
virtualisation peuvent s'interfacer facilement avec des systèmes d'annuaire
utilisés sur le marché, comme les annuaires LDAP ou ActiveDirectory (de
Microsoft), utilisés dans de nombreuses infrastructures.

\subsection{Solutions existantes}

\paragraph{} Dans cette section, nous étudierons les solutions commerciales de virtualisation d'environement reconnues pour être utilisables
dans des environnements de grande échelle.

\subsubsection{Solution VMWare}

\paragraph{} VMWare est l'éditeur de solution de virtualisation leader du
marché. VMWare propose une solution de virtualisation complète qui répond
efficacement à nos besoins. La solution intégrée proposée couvre la totalité des
besoins générés par l'adoption de ce type de technologie. La couverture de tout
les logiciels d'exploitation par l'éditeur permet d'obtenir une solution
cohérente de bout en bout entre ces différents composant. En contrepartie, cette
solution impose la dépendance à des certains protocoles propriétaires optimisés.

\paragraph{} Nous allons passer en revue les points clés de notre besoin.

\paragraph{Hébergement centralisé de machines virtuelles}

\paragraph{} \textbf{VMWare ESXi} est un hyperviseur de type bare-metal, reconnu
comme l'un des meilleurs du marché actuellement. L'avantage de ce type de
solution à OS ultra léger est double, d'une part l'optimisation du
fonctionnement de l'hyperviseur exploitant au maximum les ressources du serveur,
d'autre part une plus grande sécurité sur cette plateforme.

\paragraph{Équilibrage de charge sur les différents serveurs}

\paragraph{} \textbf{VMWare vCenter Server} offre une solution de supervision de
l'activité du parc de VM en permettant d'allouer dynamiquement les VM en
activité sur le parc de serveur, ainsi nous sommes surs de tirer parti au
maximum des ressources d'exécution des serveurs, et ce sans interruption de
service pour l'utilisateur. Nous envisageons cette solution pour les machines
dites "partagées" uniquement, par exemple les serveurs de TP entre autres.

\paragraph{Client d'accès dédié}
\paragraph{} Le client VMWare View fourni un moyen flexible d'exploiter les
différentes machines virtuelles du système.

\paragraph{} Tout d'abord, il est multi-plateforme (Windows, Linux, MacOS,
Android, autres) et cela se révèle extrêmement intéressant en terme de
limitation de coûts, car il devient alors envisageable de migrer la totalité du
parc de machines physiques  sur un système libre, l'utilisateur final étant
libre de choisir son OS sur sa VM.

\paragraph{} Pour continuer, il est possible de rapatrier et d'exécuter sur la
machine hébergeant le client les machines virtuelles, permettant d'exploiter le
parc de machines physiques présentes localement. Cela peut s'avérer extrêmement
intéressant là encore pour exploiter l'existant technique (le parc de machines)
sans avoir besoin de renouveler les serveurs. Car exécuter de façon distante les
machines virtuelles impliquerait un renouvellement du parc de serveurs.

\paragraph{Administration facilitée du parc de VM}

\paragraph{} \textbf{VMWare View Composer} Cette solution permet de générer un
ensemble (pool) de machines virtuelles "filles" héritant d'une seule et même
machine mère. Ainsi le déploiement et l'administration d'un ensemble de machines
semblables est grandement simplifiée. Cette solution, par exemple, devient utile
pour déployer une plateforme de TP. Par ailleurs, cette solution permet
d'optimiser le stockage et d'en réduire les coûts d'environ 60\% (d'après
l'éditeur).

\paragraph{} Une plateforme web est disponible pour effectuer des tâches
d'administration, elle est adaptée en fonction des droits de l'utilisateur
final.

\paragraph{Gestion des accès utilisateurs}

\paragraph{} Tous les composants de cette suite logicielle peuvent être
connectés à des systèmes d'annuaire comme LDAP ou Microsoft ActiveDirectory, qui
est actuellement utilisé par le SI de l'INSA de Lyon. Ainsi, nous serions en
mesure de fournir une gestion fine des droits utilisateurs sur le système de
machines virtuelles, par exemple au niveau de la création et l'administration de
ces dernières en se basant sur les classes d'utilisateurs actuellement définies
dans l'annuaire de l'école.

\subsubsection{Citrix Xen}

\paragraph{} Citrix Xen constitue une solution également envisageable pour notre
projet.

\paragraph{} Citrix Xen couvre le besoin fonctionnel dans sa totalité grâce à
une suite de logiciels complète. Le fonctionnement général du système de
virtualisation est similaire à celui proposé par VMWare.

\paragraph{Citrix XenServer}

\paragraph{} XenServer constitue la solution d'exécution de machines virtuelles
Xen. Il s'agit d'un hyperviseur de type bare-metal administrable à distance via
un service fourni par l'éditeur. Les performances mesurées sont cependant en
deça de la solution fournie par VMWare.

\paragraph{Citrix XenClient}

\paragraph{} Le client Xen est la solution d'accès au services de virtualisation
depuis un poste client. A la différence du client VMWare, il s'agit d'un client
de type bare-metal installé sur les postes clients. Cela demeure extrêmement
avantageux en terme de sécurité et de gestion de configuration. La
fonctionnalité d'exécution en local et distante est pleinement supportée.

\paragraph{} La couverture fonctionnelle est assurée par d'autres logiciels, qui
différent peu en terme de fonctionnalités.

\paragraph{} Cette solution est globalement plus accessible que la solution
VMWare, et demeure intéressante pour son client, qui lève toute les contraintes
d'administration des postes locaux. 

\section{État des lieux du SI du département IF}

\paragraph{} Nous estimons actuellement que le département informatique possède
une quinzaine de serveurs dédiés à la mise en place des postes de TP, 60 postes
clients complets et 60 postes déportés à l'aide de boitiers KVM et de serveurs
Blades , environ 15 postes déportés "Sun".

\paragraph{} Du coté des serveurs, le déploiement d'un système de virtualisation
nécessitera très probablement d'augmenter la puissance globale du parc de
serveurs.

\paragraph{} En revanche, il est intéressant de pouvoir exploiter les ressources
locales des machines, vu la multitude de postes physiques présents sur site.
Cela peut en revanche poser des soucis en termes d'administration, et de gestion
du volume de données qui transite par le réseau.

\paragraph{} L'infrastructure réseau actuelle supporte le trafic du déport de
l'affichage de 60 postes, le partage en réseau des documents des utilisateurs
pour environ 250 utilisateurs connectés simultanément et le trafic des
utilitaires supports (intranet, emploi du temps, etc). Cependant, la
distribution des images de machines virtuelles sur le réseau provoquerait un pic
d'utilisation lors de la connexion des utilisateurs (qui sollicitent la
plateforme au même moment lorsqu'ils arrivent en salle de TP). Il est donc
impossible de faire naviguer les images de machines virtuelles sur le réseau.
